\section{Graphical System Software for XSB}

Graphical system software for XSB is based on two systems: the
open-source InterProlog \cite{Cale01}~\footnote{InterProlog is
available through \url{www.declarativa.com}.}, and the XJ system,
which is proprietary to XSB, Inc.  While a full description of these
systems is beyond the scope of this paper, we overview some of their
important features here.

\subsection{InterProlog}

The InterProlog system provides a mechanism for XSB and Java to
intercommunicate at a finely-grained level.  Using InterProlog, a Java
message can be invoked via the predicate {\tt javaMessage} (of various
arities); and an XSB goal can be invoked via various Java messages.
From Java's point of view, Prolog is modeled by the Java class {\tt
PrologEngine}.  The extension {\tt SubprocessEngine} provides
mechanisms for XSB and Java to communicate through sockets; the
extension {\tt NativeEngine} allows XSB and Java to communicate
through Java's native interface, and it is this latter communication
that the CDF editor uses.  In either case InterProlog maps a Java
object to a Prolog term.  When a Java object is sent to Prolog {\sc
through ???}, Java's object serialization protocol is used to
translate the object to a stream of bytes.  This stream is read by
Prolog and parsed using a definite clause grammar.  When a Prolog term
is sent to Java, a definite clause grammar is again used to generate a
stream of bytes that will be read by Java.

\subsection{XJ}

InterProlog gives a powerful mechanism to write graphical front ends
for XSB, and systems such as \cite{LMC} have written sophisticated
front ends using XSB and InterProlog alone.  However, InterProlog is
not intended to address architectural questions of how Java and Prolog
should communicate within a given system.  Indeed, using InterProlog a
user could write a Java system that uses Prolog to, say 
queries

Editor Prolog 3463; XJ 12,000

\subsection{CDF Caching}

\subsection{The CDF Editor}

