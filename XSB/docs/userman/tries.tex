\chapter{Storing Facts in Tries} \label{chap:tries}
\index{tries!interned|(}

\index{indexing!trie-based}
XSB offers a mechanism by which large numbers of facts can be directly
stored and manipulated in {\em tries}.
%, which can either be private to a thread or shared among threads.
The mechanism described in this chapter is in some ways similar to
trie-indexed asserted code as described in Section~\ref{sec:assert},
%but allows creation of tries that are shared between threads,
and of associative tries that support efficient memory
management~\footnote{For nearly all purposes, the predicates in this
  chapter replace the low-level API for interned tries in previous
  versions, which included {\tt trie\_intern}, {\tt trie\_unintern},
  {\tt trie\_interned} etc.  However that API continues to be
  supported for low-level systems programming.}.

When stored in a trie, facts are compiled into trie-instructions
similar to those used for XSB's tables.  For instance set of facts
\begin{center}
\{ {\tt rt(a,f(a,b),a), rt(a,f(a,X),Y), rt(b,V,d)} \} 
\end{center}
would be stored in a trie as shown in Figure~\ref{fig:trie}, where
each node corresponds to an instruction in XSB's virtual machine.
\begin{figure}[htbp] \label{fig:trie}
\centering
\begin{tabular}{c}
\includegraphics[width=0.3\textwidth]{trie}
%%\epsfig{file=trie,height=.3\textheight}
\end{tabular}
\caption{Terms Stored as a Trie}
\end{figure} 
Using a trie for storage has the advantage that discrimination can be
made on a position anywhere in a fact, and directly inserting into or
deleting from a trie is 4-5x faster than with standard dynamic code.
In addition, in trie-dynamic code, there is no distinction between the
index and the code itself, so for many sets of facts trie storage can
use much less space than standard dynamic code.  For instance,
Figure~\ref{fig:trie} shows how the prefix {\tt rt(a,f(a,...} is
shared for the first two facts.  However, trie storage comes with
tradeoffs: first, only facts can be stored in a trie; second, unlike
standard dynamic code, no ordering is preserved among the facts; and
third, duplicate facts are not supported.

In \version{} of XSB, tries that store facts may have the following
forms:
%
\bi
\item {\em General} tries allow arbitrary terms to be inserted in a
  trie.
  %These tries are thread-private so that inserting a term in a
  %  trie $Tr$ in one thread will not be visible to another thread.
  Although such tries are general, they have limitations in memory
  reclamation in \version{} of XSB.  If a term is deleted from $Tr$,
  memory will be reclaimed if it is safe to do so at the time of
  deletion~\footnote{That is, if no choice points are around that may
    cause backtracking into $Tr$.}; otherwise the space will not be
  reclaimed until all terms in $Tr$ are removed by truncating $Tr$.
 %or   until the thread exits.

\item {\em Associative} Associative tries are more restricted
  than general tries: an associative trie combines a {\em key} which
  can be any ground term, with a {\em value} which can be any term.
  Memory for deleted key-value pairs in an associative trie is always
  immediately reclaimed, and insert or delete operations can be faster
  for an associative trie than for a general trie.
  %These tries are private to a thread, and I
  In addition to reclaiming memory when a term is deleted, memory is
  reclaimed when the trie is truncated or dropped.
%, and when the thread exits.

%\item {\em Private, general} tries allow arbitrary terms to be
%  inserted in a trie.  These tries are thread-private so that
%  inserting a term in a trie $Tr$ in one thread will not be visible to
%  another thread.  Although such tries are general, they have
%  limitations in memory reclamation in \version{} of XSB.  If a term
%  is deleted from $Tr$, memory will be reclaimed if it is safe to do
%  so at the time of deletion~\footnote{That is, if no choice points
%    are around that may cause backtracking into $Tr$.}; otherwise the
%  space will not be reclaimed until all terms in $Tr$ are removed by
%  truncating $Tr$ or until the thread exits.
%
%\item {\em Private, associative} Associative tries are more restricted
%  than general tries: an associative trie combines a {\em key} which
%  can be any ground term, with a {\em value} which can be any term.
%  Memory for deleted key-value pairs in an associative trie is always
%  immediately reclaimed, and insert or delete operations can be faster
%  for an associative trie than for a general trie.  These tries are
%  private to a thread, and in addition to reclaiming memory when a
%  term is deleted, memory is reclaimed when the trie is truncated or
%  dropped, and when the thread exits.
%
%\item {\em Shared, associative} tries are associative tries that are
%  shared among threads.  Memory for deleted key-value pairs is always
%  immediately reclaimed, and when the trie is truncated or dropped.
  \ei

\section{Examples of Using Tries}
%
A handle for a trie can be obtained using the {\tt trie\_create/2}
predicate.  Terms can then be inserted into or deleted from that trie,
and terms can be unified with information in the trie, as shown in the
following example:

\begin{example} \rm
First, we create a private general trie: 
{\small
\begin{verbatim}
| ?- trie_create(X,[type(prge)]).
X = 1

yes
\end{verbatim}
}
%
Next, we insert some terms into the trie
{\small
\begin{verbatim}
| ?- trie_insert(1,f(a,b)), trie_insert(1,[a,dog,walks]).

yes
\end{verbatim}
}
Now we can make arbitrary queries against the trie
{\small
\begin{verbatim}
| ?- trie_unify(1,X).

X = [a,dog,walks];

X = f(a,b);

no
\end{verbatim}
}
\noindent
Above, a general query was made, but the query could have been any
Prolog term.  Now we delete a term, and see what's left.
{\small
\begin{verbatim}
| ?- trie_delete(1,f(X,B)).

X = a
B = b

yes
| ?- trie_unify(1,X).

X = [a,dog,walks];

no
\end{verbatim}
}
\end{example}

The behavior of general tries can be contrasted with that of
associative tries as seen in the next example.
\begin{example} \rm
Now we create an associative trie, with abbreviation
{\tt assoc}.
%using the multi-threaded engine
%Now we start by creating a shared associative trie, with abbreviation
%{\tt shas} using the multi-threaded engine
{\small
\begin{verbatim}
| ?- trie_create(X,[type(assoc),alias(foo)]).
X = 1048577

yes
\end{verbatim}
}  \noindent
%
This time we used an alias so now we can use {\tt foo} to refer to
insert a couple of key-value pairs into the trie (we could also use
the trie handle itself) 
{\small
\begin{verbatim}
| ?- trie_insert(foo,pair(sentence(1),[a,dog,walks])), 
     trie_insert(foo,pair(sentence(2),[a,man,snores])).

yes
\end{verbatim}
} \noindent
However, inserting a general term into an associative trie throws an
error
{\small
\begin{verbatim}
| ?- trie_insert(foo,f(a,b)).
++Error[XSB/Runtime/P]: [Domain (f(a,b) not in domain pair/2)]  
in arg 2 of predicate trie_insert/2 
(Inserted term must be key-value pair in trie 1048577)
\end{verbatim}
}  \noindent
Finally, in an associative trie, if we insert a value for a key that
is already in the trie, it will {\em update} the value for that key.
{\small
\begin{verbatim}
| ?- trie_insert(foo,pair(sentence(1),[a,dog,snores])).

yes
| ?- trie_unify(foo,pair(sentence(1),X)).
X = [a,dog,snores]

yes
\end{verbatim}
}

\end{example}

\section{Space Management for Tries}~\label{sec:trie-gc}
%
When creating or adding terms to an interned trie, XSB manages all
space necessary for the terms and their indexes.  However, when
removing a term from a trie an issue may arise if there is a
possibility of backtracking into the term to be removed; this issue
also arises for retracting dynamic code.  In the sequential engine and
in private tries XSB's dynamic clause garbage collector handles space
reclamation when terms are removed from a trie through {\tt
  trie\_delete/2} or similar low-level predicates.  However, in the
case of {\tt trie\_truncate/1} or {\tt trie\_drop/1}, an exception is
thrown if there are active choice points to terms in a trie that is to
be truncated or dropped.

%In the multi-threaded engine the space reclamation problem becomes
%even more difficult for tries that can be shared among threads.  In
%this case, no garbage collection is performed until there is a single
%active thread.  

These space reclamation issues arise for non-associative tries only.
Associative triesessentially contain key-value pairs, and so may have
their space reclaimed upon deletion of a term, or upon truncation or
dropping their trie.
%, regardless of the number of active
%threads~\footnote{Future versions of XSB may extend garbage collection
%  to handle trie truncation, trie dropping and better space
%  reclamation in the multi-threaded engine.}.

\section{Predicates for Tries} 
%
The following subsections describe predicates for inserting terms into
a trie, deleting terms from a trie, and unifying a term with terms in
a trie, predicates for creating, dropping, and truncating tries, as
well as predicates for bulk inserts into and deletes from a trie.
These predicates can apply to any type of trie, and perform full error
checking on their call arguments.  As such, they are safer and more
general than the lower-level trie predicates described in Chapter 1 of
Volume 2 of this manual.  Use of the predicates described here is
recommended for applications unless the need for speed is paramount.

\begin{description}
%
\index{aliases!tries}
\ourmoditem{trie\_create(-TrieId,+OptionList)}{trie\_create/2}{intern}
%
{\tt OptionList} allows optional parameters in the configuration of a
trie to indicate its type and whether an alias should be used.  In the
present version, {\tt OptionList} may contain the following terms
\bi
\item {\tt type(Type)} where {\tt Type} can be one of
\bi
\item {\tt prge} (private, general).
%maintains information that is accessable only to the calling  thread.
  No restrictions are made for accessing information in a private
  trie.
  %In the single-threaded engine, tries are private by default.

\item {\tt pras} (private, associative) creates a private trie that
  maintains key-value pairs in a manner similar to an associative
  array, using the term {\tt pair(Key,Value)}.  Each key must be ground,
  and there may be only one value per key.

%\item {\tt shas} (shared associative) creates a shared trie that
%  maintains key-value pairs in a manner similar to an associative
%  array, using the term {\tt pair(Key,Value)}.  Each key must be
%  ground, and there may be only one value per key.  This option is
%  available only in the multi-threaded engine

%\item {\tt shared(full)} Maintains information that is accessable to
%  all threads, with no restrictions on its access.  In the
%  single-threaded engine, there is no distinction between private and
%  shared tries.
\ei
\item {\tt alias(Alias)}: Allow trie {\tt TrieId} to be referred to
  via {\tt Alias} in all standard trie predicates.  {\tt Alias}
  remains active for {\tt TrieId} until it is dropped.
%
\item {\tt incremental}: Allows tables that depend on trie {\tt
  TrieId} to be automatically updated as information in {\tt TrieId}
  changes (cf. Section~\ref{sec:incr-update-tries}).
%
\item {\tt nonincremental}: Specifies that tables that depend on trie
  {\tt TrieId} should not be automatically updated as information in
  {\tt TrieId} changes (cf. Section~\ref{sec:incr-update-tries}).
\end{itemize}
%
{\bf Error Cases}
\bi
\item 	{\tt TrieId} is not a variable
\bi
\item 	{\tt type\_error(variable,TrieId)}
\ei
\item 	{\tt OptionList} is a partial list or contains an option that is a variable
\bi
\item 	{\tt instantiation\_error}
\ei
\item 	{\tt OptionList} is neither a list nor a partial list
\bi
\item 	{\tt type\_error(list,OptionsList)}
\ei
\item 	{\tt OptionList} contains an option, {\tt Option} not described above
\bi
\item 	{\tt domain\_error(trie\_option,Option)}
\ei
\item An element of {\tt OptionsList} is alias(A) and A is already
  associated with an existing %thread,
  queue, mutex or stream 
\bi
\item {\tt permission\_error(create,alias, A)}
\ei
\item An element of {\tt OptionsList} is alias(A) and A its not an atom
\bi
\item {\tt type\_error(atom,A)}
\ei
\ei
\index{tabling!incremental}

\index{terms!cyclic}
\ourmoditem{trie\_insert(+TrieIdOrAlias,Term)}{trie\_insert/2}{intern}
%
Inserts {\tt Term} into the trie denoted by {\tt TrieIdOrAlias}.  If
{\tt TrieIdOrAlias} denotes an associative trie, {\tt Term} must be of
the form {\tt pair(Key,Value)} where {\tt Key} is ground.  If {\tt
  TrieIdOrAlias} is a general trie and already contains {\tt Term},
the predicate fails (as the same term cannot be inserted multiple
times in the same trie).  Similarly, if {\tt TrieIdOrAlias} is an
associative trie and already contains a value for {\tt Key} the
predicate fails.

Insertion of tries can be controlled by the flags {\tt
  max\_answer\_term\_depth}, {\tt max\_answer\_list\_depth}, {\tt
  max\_answer\_term\_action}, and {\tt max\_answer\_list\_action},
which are also used to control additions of answers to tables.  Using
these flags, if a term to be inserted is cyclic and exceeds a stated
depth, trie insertion may either fail or throw an error depending on
the associated action: see pg. \pageref{prolog-flags}.

{\bf Error Cases}
\bi
\item 	{\tt TrieIdOrAlias} is a variable
\bi
\item 	{\tt instantiation\_error}.
\ei
\item 	{\tt TrieIdOrAlias} is not a trie id or alias
\bi
\item 	{\tt domain\_error(trie\_id\_or\_alias,TrieIdOrAlias)}
\ei
\item 	{\tt TrieIdOrAlias} denotes an associative array, and {\tt Term} 
  does not unify with {\tt pair(\_,\_)} 
\bi
\item 	{\tt domain\_error(pair/2,Term)}
\ei
\item 	{\tt TrieIdOrAlias} denotes an associative array, 
  {\tt Term = pair(Key,Value)} but {\tt Key} is not ground 
\bi
\item 	{\tt misc\_error}
\ei
\item {\tt Key} or {\tt Value} is a cyclic term, or exceeds the depth 
\bi
\item 	{\tt misc\_error}
\ei
\ei
%

\ourmoditem{trie\_unify(+TrieIdOrAlias,Term)}{trie\_unify/2}{intern}
%
Unifies {\tt Term} with a term in the trie denoted by {\tt
  TrieIdOrAlias}.  If {\tt TrieIdOrAlias} denotes a general trie,
successive unifications will succeed upon backtracking.  If {\tt
  TrieIdOrAlias} denotes an associative trie, {\tt Term} must be of
the form {\tt pair(Key,Value)} where {\tt Key} is ground.

{\bf Error Cases}
\bi
\item 	{\tt TrieIdOrAlias} is a variable
\bi
\item 	{\tt instantiation\_error}.
\ei
\item 	{\tt TrieIdOrAlias} is not a trie id or alias
\bi
\item 	{\tt domain\_error(trie\_id\_or\_alias,TrieIdOrAlias)}
\ei
\item 	{\tt TrieIdOrAlias} denotes an associative array, and {\tt Term} 
  does not unify with {\tt pair(\_,\_)} 
\bi
\item 	{\tt domain\_error(pair/2,Term)}
\ei
\item {\tt TrieIdOrAlias} denotes an associative array, 
  {\tt Term = pair(Key,Value)} but {\tt Key} is not ground 
\bi
\item 	{\tt misc\_error}
\ei
\ei

\ourmoditem{trie\_delete(+TrieIdOrAlias,Term)}{trie\_delete/2}{intern}
%
Deletes a term unifying with {\tt Term} from the trie denoted by {\tt
  TrieIdOrAlias}.  {\tt TrieIdOrAlias} denotes a general trie, all
such terms can be deleted upon backtracking.  If {\tt TrieIdOrAlias}
denotes an associative trie, {\tt Term} must be of the form {\tt
  pair(Key,Value)} where {\tt Key} is ground.  In either case, if {\tt
  TrieIdOrAlias} does not contain a term unifying with {\tt Term} the
predicate fails.

{\bf Error Cases}
\bi
\item 	{\tt TrieIdOrAlias} is a variable
\bi
\item 	{\tt instantiation\_error}.
\ei
\item 	{\tt TrieIdOrAlias} is not a trie id or alias
\bi
\item 	{\tt domain\_error(trie\_id\_or\_alias,TrieIdOrAlias)}
\ei
\item 	{\tt TrieIdOrAlias} denotes an associative array, and {\tt Term} 
  does not unify with {\tt pair(\_,\_)} 
\bi
\item 	{\tt domain\_error(pair/2,Term)}
\ei
\item {\tt TrieIdOrAlias} denotes an associative array, 
  {\tt Term = pair(Key,Value)} but {\tt Key} is not ground 
\bi
\item 	{\tt misc\_error}
\ei
\ei
%
\ourmoditem{trie\_truncate(+TrieIdOrAlias)}{trie\_truncate/1}{intern}
%
Removes all terms from {\tt TrieIdOrAlias}, but does not change any of
its properties (e.g. the type of the trie or its aliases).  

@@

{\bf Error Cases}
\bi
\item 	{\tt TrieIdOrAlias} is a variable
\bi
\item 	{\tt instantiation\_error}.
\ei
\item 	{\tt TrieIdOrAlias} is not a trie id or alias
\bi
\item 	{\tt domain\_error(trie\_id\_or\_alias,TrieIdOrAlias)}
\ei
\item There are active failure continuations to terms in {\tt
  TrieIdOrAlias}
\bi
\item 	{\tt miscellaneous\_error}
\ei
\ei

\ourmoditem{trie\_drop(+TrieIdOrAlias)}{trie\_drop/1}{intern}
%
Drops {\tt TrieIdOrAlias}.  {\tt trie\_drop/1} not only removes all
terms from {\tt TrieIdOrAlias}, but also removes information about its
type and any aliases the trie may have.

{\bf Error Cases}
\bi
\item 	{\tt TrieIdOrAlias} is a variable
\bi
\item 	{\tt instantiation\_error}.
\ei
\item 	{\tt TrieIdOrAlias} is not a trie id or alias
\bi
\item 	{\tt domain\_error(trie\_id\_or\_alias,TrieIdOrAlias)}
\ei
\item There are active failure continuations to terms in {\tt
  TrieIdOrAlias}
\bi
\item 	{\tt miscellaneous\_error}
\ei
 \ei

\ourmoditem{trie\_bulk\_insert(+TrieIdOrAlias,+Generator)}{trie\_bulk\_insert/2}{intern}
% 
Used to insert multiple terms into the trie denoted by {\tt
  TrieIdOrAlias}.  {\tt Generator} must be a callable term.  Upon
backtracking through {\tt Generator} its first argument should
successively be instantiated to the terms to be interned in {\tt
  TrieIdOrAlias}.  When inserting many terms into a general trie, {\tt
  trie\_bulk\_insert/2} is faster than repeated calls to {\tt
  trie\_insert/2} as it does not need to make multiple checks that the
choice point stack is free of failure continuations that point into
the {\tt TrieIdOrAlias} trie.  For associative tries, {\tt
  trie\_bulk\_insert/2} can also be faster as it needs to perform
fewer error checks on the arguments of the insert.

\begin{example} \rm
Given the predicate 
\begin{verbatim}
bulk_create(p(One,Two,Three),N):- 
     for(One,1,N),
     for(Two,1,N),
     for(Three,1,N).
\end{verbatim}
and a general trie {\tt Trie}, the goal 
\begin{center}
   {\tt ?- trie\_bulk\_insert(Trie,bulk\_create(\_Term,N))} 
\end{center}
will add $N^3$ terms to {\tt Trie}.
\end{example}

{\bf Error Cases}
\bi
\item 	{\tt TrieIdOrAlias} is a variable
\bi
\item 	{\tt instantiation\_error}.
\ei
\item 	{\tt TrieIdOrAlias} is not a trie id or alias
\bi
\item 	{\tt domain\_error(trie\_id\_or\_alias,TrieIdOrAlias)}
\ei
\item   {\tt Generator} is not a compound term
\bi
\item   {\tt type\_error(compound,Generator)}
\ei
\item 	{\tt TrieIdOrAlias} denotes an associative array, and {\tt Generator} 
  does not unify with {\tt pair(\_,\_)} 
\bi
\item 	{\tt domain\_error(pair/2,Term)}
\ei
\item {\tt TrieIdOrAlias} denotes an associative array, and {\tt
  Generator} succeeds with a term that unifies with {\tt
  pair(Key,Value)} and {\tt Key} is not ground 
\bi
\item 	{\tt misc\_error}
\ei
\item {\tt Key} or {\tt Value} is a cyclic term
\bi
\item 	{\tt misc\_error}
\ei
\ei

\index{terms!cyclic}
\ourmoditem{trie\_bulk\_delete(+TrieIdOrAlias,Term)}{trie\_bulk\_delete/2}{intern}
% 
Deletes all terms that unify with {\tt Term} from {\tt TrieIdOrAlias}.
If {\tt TrieIdOrAlias} denotes an associative trie, the key of the key
value pair need {\em not} be ground.

\begin{example}\label{ex:bulk-delete} \rm
For the trie in the previous example, the goal 
\begin{center}
{\tt ?-  trie\_bulk\_delete(Trie,p(1,\_,\_))} 
\end{center}
will delete the $N^2$ terms that unify with {\tt p(1,\_,\_)} from {\tt TrieIdOrAlias}.
\end{example}

{\bf Error Cases}
\bi
\item 	{\tt TrieIdOrAlias} is a variable
\bi
\item 	{\tt instantiation\_error}.
\ei
\item 	{\tt TrieIdOrAlias} is not a trie id or alias
\bi
\item 	{\tt domain\_error(trie\_id\_or\_alias,TrieIdOrAlias)}
\ei
\ei

\ourmoditem{trie\_bulk\_unify(+TrieIdOrAlias,\#Term,-List)}{trie\_bulk\_unify/3}{intern}
% 
Returns in {\tt List} all terms in {\tt TrieIdOrAlias} that unify with
{\tt Term}.  If {\tt TrieIdOrAlias} denotes an associative trie, the
key of the key value pair need {\em not} be ground.

This predicate is useful for two reasons.  First, it provides a safe
way to backtrack through an associative trie while maintaining the
memory management and concurrency properties of associative tries.
Second, it enforces read consistency for {\tt TrieIdOrAlias},
regardless of whether the trie is private or shared, general or
associative.

\begin{example} \rm
Continuing from Example~\ref{ex:bulk-delete} the goal
\begin{center}
{\tt ?-  trie\_bulk\_unify(Trie,X),List} 
\end{center}
will return the the $N^3 - N^2$ terms still in {\tt TrieIdOrAlias}.
\end{example}

{\bf Error Cases}
\bi
\item 	{\tt TrieIdOrAlias} is a variable
\bi
\item 	{\tt instantiation\_error}.
\ei
\item 	{\tt TrieIdOrAlias} is not a trie id or alias
\bi
\item 	{\tt domain\_error(trie\_id\_or\_alias,TrieIdOrAlias)}
\ei
\item 	{\tt List} is not a variable
\bi
\item 	{\tt type\_error(variable,List)}.
\ei
\ei

\ourmoditem{trie\_property(?TrieOrAlias,?Property)}{trie\_property/2}{intern}
%
If {\tt TrieOrAlias} is instantiated, unifies {\tt Property} with
current properties of the trie; if {\tt TrieOrAlias} is a variable,
backtracks through all the current tries whose properties unify with
{\tt Property}.
%In the MT engine, {\tt thread\_property/2} accesses
%only tries private to the calling thread and shared tries; however
%note that there is no guarantee that that the information returned
%about shared tries will be valid, due to concurrency
%issues~\footnote{{\tt trie\_property/2} is not yet implemented for
%  shared tries.}.

Currently {\tt Property} can have the form 
\bi
\item {\tt type(Type)}: where {\tt Type} is the type of the trie.
%
\item {\tt alias(Alias)}: if the trie has an alias {\tt Alias}
\ei

{\bf Error Cases}
%
\bi
\item {\tt TrieOrAlias} is neither a variable nor an XSB trie id
  nor an alias
\bi
\item {\tt domain\_error(trie, TrieOrAlias)}
\ei
\item {\tt TrieOrAlias} is not associated with a valid trie
\bi
\item {\tt existence\_error(trie, TrieOrAlias)}
\ei
\ei

%\ouritem{trie\_property(+TrieIdOrAlias,Property)}
%\index{\texttt{trie\_property/2}}
%
\end{description}

\section{Low-level Trie Manipulation Utilities}

%The previous sections indicate how tries can be used as an efficient
%mechanism to store thread-private and thread-shared terms.
In this section we describe lower-level trie manipulation predicates
that are suitable for implementing XSB libraries~\footnote{Flora-2,
  XASP, XSB's {\tt storage} library and others use these predicates.}.
As with other tries, these utilities are suitable for storing terms
rather than executable clauses, use a set based semantics, and do not
maintain an ordering among these terms.  In addition
%
\bi
\item These predicates create and maintain
%thread-private,
  general tries.
%
\item These predicates do not always perform error checking.  If not
  explicitly specified in the description of the predicate, errors
  returned may be confusing, and calling with improper arguments may
  even cause memory violations.  
%
\item For historical reasons, the ordering of arguments in these
  predicates is not consistent.
\ei
%
Despite (and sometimes because of) these limitations, the trie
manipulation facilities can be extremely fast, so that interning and
uninterning terms in a trie may be much faster than assert and retract
in XSB or in any other Prolog.

\subsection{A Low-Level API for Interned Tries} \label{sec:intern-basic}
%%
\begin{description}
\ourmoditem{new\_trie(-Root)}{new\_trie/1}{intern} {\tt Root} is
instantiated to a handle for a new private, general trie.


\ourrepeatmoditem{trie\_intern(+Term,+Root)}{trie\_intern/2}{intern}
%
\ourmoditem{trie\_intern(+Term,+Root,-Leaf,-Flag,-Skel)}{trie\_intern/5}{intern}
% 
{\tt trie\_intern/2} effectively asserts \texttt{Term} by interning
into the trie designated by {\tt Root}.  If a variant of {\tt Term} is
already in {\tt Root} the predicate succeeds, but a new copy of {\tt
  Term} is not added to the trie.

{\tt trie\_intern/5} acts as {\tt trie\_intern/2} but returns
additional information: {\tt Leaf} is the handle for the interned {\tt
  Term} in the trie.  {\tt Flag} is 1 if the term is ``old'' (already
exists in the trie); it is 0, if the term is newly inserted.  {\tt
  Skel} represents the collection of all the variables in {\tt
  Term}. It has the form {\tt ret(V1,V2,...,VN)}, exactly as in {\tt
  get\_calls} (see Vol. 1 of the XSB manual).

{\bf Error Cases}
\begin{itemize}
\item 	{\tt Root} is uninstantiated
\bi
\item 	 {\tt instantiation\_error}
\ei
\item 	{\tt Root} is instantiated, but not an integer (trie handle)
\bi
\item 	 {\tt type\_error(integer,Root)}
\ei
\end{itemize}
%%

\ourrepeatmoditem{trie\_interned(?Term,+Root)}{trie\_interned/2}{intern}
%%
\ourmoditem{trie\_interned(?Term,+Root,?Leaf,-Skel)}{trie\_interned/4}{intern}
%%
{\tt trie\_interned/2} backtracks through the terms that unify with
{\tt Term} and that are interned into the trie represented by the
handle {\tt Root}.  {\tt Term} may be free, or partially bound.

If {\tt Leaf} is a free variable, {\tt trie\_interned/5} works as {\tt
  trie\_interned/2}: it backtracks through the terms that unify with
{\tt Term} interned into the trie represented by the handle {\tt
  Root}.  In addition it returns {\tt Leaf} as the handle for each
such term and returns in {\tt Skel} the collection of all the
variables in {\tt Term} using the form {\tt ret(V1,...,Vn)}.
Otherwise, if {\tt Leaf} is bound, {\tt trie\_interned/5} will unify
{\tt Term} with the term in the trie designated by {\tt Leaf}, 
returning a vector of variables in {\tt Skel}.

{\bf Error Cases}
\begin{itemize}
\item 	{\tt Root} is uninstantiated
\bi
\item 	 {\tt instantiation\_error}
\ei
\item 	{\tt Root} is instantiated, but not an integer (trie handle)
\bi
\item 	 {\tt type\_error(integer,Root)}
\ei
\end{itemize}
%%

\ourrepeatmoditem{trie\_unintern(+Root,+Leaf)}{trie\_unintern2}{intern}
\ourmoditem{trie\_unintern\_nr(+Root,+Leaf)}{trie\_unintern\_nr/2}{intern}
%%
{\tt trie\_unintern(+Root,+Leaf)} deletes a term from a trie using the
handle {\tt Leaf}, as obtained from {\tt trie\_intern/[2,4]} or {\tt
  trie\_interned/[2,4]}.  Space is reclaimed for the term only if it
is safe to do so -- if there are no failure continuations that may
consume the term (cf. Section~\ref{sec:trie-gc}).

{\tt trie\_unintern\_nr/2} does not perform space reclamation and as a
result requires no garbage collection -- it simply marks a term as
``deleted''.  This makes {\tt trie\_unintern\_nr/2} suitable if trie
garbage collection may be an issue, and also allows it to be used in
libraries that support backtrackable updates, such as XSB's {\tt
  storage} library.

{\bf Error Cases}
\begin{itemize}
\item 	{\tt Root} or {\tt Leaf} is uninstantiated
\bi
\item 	 {\tt instantiation\_error}
\ei
\item 	{\tt Root} or {\tt Leaf} is instantiated, but not an integer
  (trie handle or trie leaf) 
\bi
\item 	 {\tt type\_error(integer,Root)} or {\tt type\_error(integer,Leaf)}
\ei
\end{itemize}
%%

\ourmoditem{reclaim\_uninterned\_nr(+Root)}{reclaim\_uninterned\_rn/1}{intern}
%%
Runs through the chain of leaves of the trie {\tt Root} and deletes
the terms that have been marked for deletion by {\tt
  trie\_unintern\_nr/2}. This can be viewed either as a garbage
collection step or as a commit.

{\bf Error Cases}
\begin{itemize}
\item 	{\tt Root} is uninstantiated
\bi
\item 	 {\tt instantiation\_error}
\ei
\item 	{\tt Root} is instantiated, but not an integer (trie handle)
\bi
\item 	 {\tt type\_error(integer,Root)}
\ei
\end{itemize}
\ourmoditem{unmark\_uninterned\_nr(+Root,+Leaf)}{unmark\_uninterned\_nr/2}{intern}
The term pointed to by {\tt Leaf} should have been previously marked
for deletion using {\tt trie\_unintern\_nr/2}. This term is then
``unmarked'' (or undeleted) and becomes again a normal interned term.

{\bf Error Cases}
\begin{itemize}
\item 	{\tt Root} or {\tt Leaf} is uninstantiated
\bi
\item 	 {\tt instantiation\_error}
\ei
\item 	{\tt Root} or {\tt Leaf} is instantiated, but not an integer
  (trie handle or trie leaf) 
\bi
\item 	 {\tt type\_error(integer,Root)} or {\tt type\_error(integer,Leaf)}
\ei
\end{itemize}
%%
\ourmoditem{delete\_trie(+Root)}{delete\_trie/1}{intern} 
%%
Deletes all the terms in the trie pointed to by {\tt Root}.  Garbage
collection ensures that space reclamation is performed only if it is
safe to do so.

{\bf Error Cases}
\begin{itemize}
\item 	{\tt Root} is uninstantiated
\bi
\item 	 {\tt instantiation\_error}
\ei
\item 	{\tt Root} is instantiated, but not an integer (trie handle)
\bi
\item 	 {\tt type\_error(integer,Root)}
\ei
\item 	Failure continuations point to one or more nodes in the trie with root {\tt Root}
\bi
\item 	{\tt misc\_error}
\ei
\end{itemize}
%%

\end{description}

\index{tries!interned|)}
